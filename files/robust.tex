%TODO: example Hinfty
%TODO: old parameter K stuff?

\section{Robust Control}

\subsection{Norms for signals and systems}
The $\mathcal{L}_2$ (euclidian norm) of continuous signal $u$ or a vector-valued signal $\mathbf{u}$ is
\begin{align*}
    ||u||_2 &= \sqrt{\int_{-\infty}^{\infty} u^2(t)\, dt} &
    ||\mathbf{u}||_2 = \sqrt{\int_{-\infty}^{\infty} \mathbf{u}^T \mathbf{u}\, dt}
\end{align*}

The $\mathcal{H}_2$ norm of a \emph{stable} system $G(s)$ is defined as the $\mathcal{L}_2$
norm of its impulse response $g(t)$.

The $\mathcal{H}_{\infty}$ norm is defined as the maximum RMS amplification over all arbitrary input signals $u\neq 0$.
This equals the peak magnitude in the Bode diagram of the transfer function $G(j\omega)$.
\[
    ||G||_{\infty} = \sup_{u \in \mathcal{L}_2} \frac{||y||_2}{||u||_2} = \sup_{\omega} |G(j\omega)|
\]

MIMO: $\left|\left|G\right|\right|_{\infty}=\underset{\omega}{\mathrm{max}}\,\bar{\sigma}\big(G(j\omega)\big)$ (max of largest singular value $\overline{\sigma}$)

SISO: $||G||_{\infty}=\operatorname*{max}_{\omega}|G(j\omega)|$ (max value of $|G(j\omega)|$)
\hspace{0.5cm}

Cauchy-Schwarz-inequality: (serial $G$s)

$$||G_{1}\cdot G_{2}||_{\infty}\leq ||G_{1}||_{\infty}\cdot ||G_{2}||_{\infty}$$

parallel $G$s

$$\left| \left| \begin{bmatrix}G_{1} \\ Gs\end{bmatrix} \right| \right|_{\infty}\leq 1\Rightarrow||G_{1}||_{\infty}\leq 1,\  ||G_{2}||_{\infty}\leq 1$$


\subsection{The Nyquist Criterion and the Small Gain Theorem}

\begin{minipage}{10cm}
    \paragraph{Nyquist Criterion}The closed loop system with loop gain $L(s)$ and a negative feedback polarity
    is stable if the Nyquist plot of $L(j\omega)$ encircles the critical point $s_{crit}=-1$ exactly
    $N_P$ times anticlockwise, where $N_P$ is the number of unstable poles of $L(s)$.
    
    If $L(s)$ is a stable open loop transfer function, $N_P = 0$.

    \paragraph{Gain Margin, Phase Margin and Critical Distance}
    In robust control, the critical distance $d_{crit}$ is used instead
    of phase and gain margin. 
    \[
        d_{crit} = \min_{\omega} |L(j\omega) - s_{crit}| = \min_{\omega}|1+L(j\omega)|
    \]
    This is the reciprocal of the sensitivity function
    \[
        d_{crit} = \frac{1}{||S||_{\infty}}, \quad \text{where } S(j\omega) = \frac{1}{1 + L(j\omega)}
    \]
\end{minipage}
\hspace{0.5cm}
\begin{minipage}{8cm}
    \centering
    \includegraphics[width=\linewidth]{bilder/robust_nyquist.png}
\end{minipage}

\paragraph{Small Gain Theorem}
If $L(s)$ is stable and $||L||_{\infty} < 1$, then the loop gain $|L(j\omega)|$ is smaller
than $1$ for all frequencies $\omega$.
Hence, $L(j\omega)$ does not encircle the critical point $-1$.
The resulting phase margin is \emph{infinite}: $\phi_m = \infty$.

The small gain condition $||L||_{\infty} < 1$ is \emph{sufficient} but not \emph{necessary}
for closed loop stability.

\begin{minipage}{10cm}
    \paragraph{Application of the Small Gain Theorem}
    For a plant $P(s)$ with a nominal transfer function $P_0(s)$, an additive 
    perturbation $\Delta_a(s)$, and a stabilizing controller $C(s)$, the 
    question arises, how large $\Delta_a(s)$ can become
    before becoming unstable.
    
    The feedback seen by $\Delta_a$ is $\frac{-G_C}{1+G_{p0} G_C}$.
    A sufficient condition for the stability using the small gain theorem is therefore
    \begin{align*}
        \left\lVert \Delta_a \cdot \frac{-G_C}{1+G_{p0} G_C}\right\rVert_{\infty} < 1 && \Leftrightarrow &&
        ||\Delta_a||_{\infty} < \frac{1}{\left\lVert \frac{-G_C}{1+G_{p0} G_C} \right\rVert_{\infty}}\\
        && \Leftrightarrow &&||\Delta_a||_{\infty} <\frac{1}{\left\lVert-G_CS\right\rVert_{\infty}}
    \end{align*}
\end{minipage}
\hspace{0.5cm}
\begin{minipage}{8cm}
    \centering
    \includegraphics[width=7cm]{bilder/rob_smallgain (2).png}
\end{minipage}

The Bode integral theorem states that
\[
    \int_{0}^{\infty} \ln |S(j\omega)|\, d\omega = \pi \sum_{i} \Re(p_i)-\frac{\pi}{2}\lim_{s\to \infty}\cdot L(s), \quad \text{unstable poles $p_i$ of $L(s)$}
\]

\subsection{Unstructured Uncertainty}

\begin{minipage}[t]{0.49\linewidth}
    \paragraph{Additive Uncertainty}
    \begin{center}
        \includegraphics[height=1.5cm]{bilder/rob_add (2).png}
    \end{center}
    Transfer function:
    \[
        G_P(s) = G_{p0}(s) + \Delta_a(s)
    \]
    and
    \[
        \Delta_a(s) = W_{2a}(s) \cdot \tilde{\Delta}_a(s)
    \]
    where $\tilde{\Delta}_a$ is \emph{any} stable, normalized perturbation with
    \[
        ||\tilde{\Delta}_a||_{\infty} \leq 1
    \]
    and the radius is scaled with $W_{2a}(s)$.
    
    From the small gain theorem, the perturbed system is stable if
    \[
        ||\Delta_a||_{\infty} < \frac{1}{\left\lVert\frac{G_C}{1+G_{p0}G_C}\right\rVert_{\infty}}=\frac{1}{\left\lVert G_CS\right\rVert_{\infty}}
    \]
\end{minipage}
\begin{minipage}[t]{0.49\linewidth}
    \paragraph{Multiplicative Uncertainty}
    \begin{center}
        \includegraphics[height=1.5cm]{bilder/rob_mult (2).png}
    \end{center}
    Transfer function:
    \[
        G_p(s) = G_{p0}(s) \cdot (1 + \Delta_m(s))
    \]
    and
    \[
        \Delta_m(s) = W_{2m}(s) \cdot \tilde{\Delta}_m(s)
    \]
    where $\tilde{\Delta}_m$ is \emph{any} stable, normalized perturbation with
    \[
        ||\tilde{\Delta}_m||_{\infty} \leq 1
    \]
    and the radius is scaled with $W_{2m}(s)$.
    
    From the small gain theorem, the perturbed system is stable if
    \[
        ||\Delta_m||_{\infty} < \frac{1}{\left\lVert\frac{G_{p0}G_C}{1+G_{p0}G_C}\right\rVert}=\frac{1}{\left\lVert T\right\rVert_{\infty}}
    \]
\end{minipage}

\vspace{0.5em}

In the SISO case, additive uncertainty can be recast into multiplicative
uncertainty by
\[
    G_{p0}(s) \Delta_m(s) = \Delta_a(s)
\]

\paragraph{Dead time}An uncertain dead-time can be described as multiplicative uncertainty
\begin{align*}
    G_p(s) = G_{p0}(s) \cdot e^{-sT} && \Leftrightarrow && \Delta_m(s) = e^{-sT} - 1
\end{align*}

\subsection{Linear Fractional Transformation (LFT)}
\begin{minipage}{12cm}
    The plant can be partitioned according to the dimensions of $\Delta$
    \[
        P_g = \begin{bmatrix}
            P_{11} & P_{12} \\
            P_{21} & P_{22}
        \end{bmatrix}
    \]

    which corresponds to
    \[
        e = \left( P_{22} + P_{21} \Delta(I-P_{11}\Delta)^{-1} P_{12} \right) u
    \]
    We define the upper \emph{linear fractional transformation} (LFT) of $P$ and $\Delta$ as
    \[
        \mathcal{F}(P_g,\Delta) = P_{22} + P_{21} \Delta (I-P_{11}\Delta)^{-1} P_{12}
    \]
    an the lower LFT is
    \[
        \mathcal{F}(P_g,G_C) = P_{11} + P_{12} G_C (I-P_{22}G_C)^{-1} P_{21}
    \]

\end{minipage}
\hspace{0.5cm}
\begin{minipage}{6cm}
    \centering
    \includegraphics[width=4cm]{bilder/rob_lftupper.png}

    \vspace{1cm}

    \includegraphics[width=4cm]{bilder/rob_lftlower.png}
\end{minipage}



Additive and multiplicative uncertainties are special cases of the LFT:
\begin{align*}
    \text{Additive:} \quad
    P_{ga} = \begin{bmatrix}
        0 & I \\ I & G_{p0}
    \end{bmatrix}
    &&
    \text{Multiplicative:} \quad
    P_{gm} = \begin{bmatrix}
        0 & I \\ G_{p0} & G_{p0}
    \end{bmatrix}
\end{align*}

\subsection{Structured Uncertainty}
Concrete physical parameters and tolerances can be modeled using structured uncertainties.
A structured uncertainty leads to a \emph{static diagonal} block $\Delta$. The diagonal
elements are \emph{real-valued}. The diagonal elements of \(\Delta\) are physical parameters.
(In contrast to mostly complex-valued, norm-bounded transfer functions $\Delta(s)$ as in unstructured uncertainties.)

\subsection{\texorpdfstring{The Standard $\bm{\mathcal{H}}_{\bm{\infty}}$ Problem}{The Standard H-inf Problem}}
The $\mathcal{H}_{\infty}$ design is to create robust controllers. 
A system which is stable with all possible uncertainties of a plant is called
\emph{robustly stable}.

\[T_{zw}=F_l(P_{aug},G_C)\Rightarrow||T_{zw}||_\infty\max_{w\neq0}\frac{||z||_2}{||w||_2}\]

Find the $G_C$ that minimizes \(||T_{zw}||_\infty\)

\begin{minipage}{12cm}
    The augumeted plant perturbed with \(\Delta\), exogenous inputs \(w\) \((r, d, \dots)\) and exogenous outputs \(z\) \(e, u, \dots\).
    Find the sabilizing controller \(G_C\), where \(T_{zw}\) is given by
    \[
        T_{z w}=F_{u}(F_{l}(P_{aug},G_{c}),\Delta)
    \]
    under the worst-case \(\Delta\)
    \[
        ||T_{z w}||_{\infty}=\max_{w\neq0}\frac{||z||_{2}}{||w||_{2}}\qquad\qquad\min_{C}(\max_{\Delta}||T_{z w}||_{\infty})
    \]


\end{minipage}
\hspace{0.5cm}
\begin{minipage}{6cm}
    \centering
    \includegraphics[width=4cm]{bilder/rob_lftul.png}
\end{minipage}

\subsection{Performance specification}
Minimizing \(||S||_\infty\) ensures robustness, but no good disturbance rejection at low frequencies (bad performance).
Minimizing \(||T||_\infty\) avoids resonance behavior at cross-over, but does neither lead to good tracking nor to robustness. 

Instead of directly Minimizing \(S\) introduce a weighting function \(W_1S\).
\[
    \min_{C}(||S||_{\infty})\Rightarrow\min_{C}(||W_{1}S||_{\infty})
\]
\(W_{1}S\) is used to shape \(S\)
\[
    W_{1}(s)=\hat{S}(s)^{-1}
\]
Design a weighting function to shape \(S\)
Matlab \dq W = makeweight(GDC,wB,GHF,n); \dq{}

\subsection{Mixed Sensitivity}
% From the small gain theorem, it follows that a system is robustly stable, if
\begin{minipage}{10cm}
    \begin{align*}
        \text{Additive:}\quad&
        ||W_{2a}G_CS||_{\infty} \leq 1 \\
        \text{Multiplicative:}\quad&
        ||W_{2m}T||_{\infty} \leq 1
    \end{align*}
    combined with Performance weighting
    \[
        ||W_{1}S||_{\infty}< 1
    \]
    results in 
    \[
    \begin{Vmatrix}
        W_{1}S\\
        W_{2a}G_CS
    \end{Vmatrix}_\infty<1
    \qquad\qquad
    \begin{Vmatrix}
        W_{1}S\\
        W_{2m}T
    \end{Vmatrix}_\infty<1
    \]
\end{minipage}
\hspace{0.5cm}
\begin{minipage}{8cm}
    \centering
    \includegraphics[width=8cm]{bilder/rob_hinf_w.png}
\end{minipage}

% Further, the sensitivity function $S=(I+PC)^{-1}$ must meet the condition
% \begin{align*}
%     || \gamma W_1 (I+PC)^{-1} ||_{\infty} \leq 1 
%     && \Rightarrow &&
%     S(j\omega) \leq \frac{1}{|W_1(j\omega)|}
% \end{align*}
% where $\gamma$ is an optimization parameter. 
% The larger $\lambda$, the better the disturbance attenuation.
% A controller is now created by solving the optimization problem:
% \begin{align*}
%     \text{Additive:}\quad&
%     \max_{\lambda} \left\lVert \begin{bmatrix}
%         \lambda W_1(s) (I+PC)^{-1} \\
%         W_{2a}(s) C (I+PC)^{-1}
%     \end{bmatrix} \right\rVert_{\infty} = 1
%     \\
%     \text{Multiplicative:}\quad&
%     \max_{\lambda} \left\lVert \begin{bmatrix}
%         \lambda W_1(s) (I+PC)^{-1} \\
%         W_{2m}(s) PC (I+PC)^{-1}
%     \end{bmatrix} \right\rVert_{\infty} = 1
% \end{align*}

\vspace{0.5cm}

This is equivalent to finding a controller $C$, for which
\[
    ||\mathcal{F}(P_{aug},G_C)||_{\infty} = ||P_{11} + P_{12}G_C(I-P_{22}G_C)^{-1} P_{21}||_{\infty} = 1
\]

\subsection{\texorpdfstring{$\bm{\mathcal{H}}_{\bm{\infty}}$ Synthesis}{H-inf Synthesis}}

The optimization solution is not unique. 
In practical design it is sufficient to find a stabilizing \(G_C\) such that \(||T_{zw}||_\infty\) is less than a samm positive number
\[
    ||T_{z w}||_{\infty}=F_{l}(P_{a u g},G_{C})< \gamma
\]
This is the suboptimal solution of \(H_\infty\) because
\[
    \gamma > \gamma_{0}=\min_{C}||F_{l}(P_{a u g},G_{C}||_{\infty}
\]
results in 
\[
\begin{Vmatrix}
    W_{1}S\\
    W_{2a}G_CS
\end{Vmatrix}_\infty<\gamma
\qquad\qquad
\begin{Vmatrix}
    W_{1}S\\
    W_{2m}T
\end{Vmatrix}_\infty<\gamma
\]

\paragraph{State Space Solution}~\\
\begin{minipage}{10cm}
    The state space representation is
    \begin{align*}
        \dot{x} &= Ax + B_1 w + B_2 u \\
        z &= C_1 x + D_{11} w + D_{12} u \\
        e &= C_2 x + D_{21}w + D_{22} u
    \end{align*}
    
    with the general system matrix
    \[
        \Rightarrow \qquad
        \begin{bmatrix}
            \dot{x}\\
            z\\
            e
        \end{bmatrix}=
        \underbrace{\begin{bmatrix}
            A & \vdots & B_1 & B_2 \\
            \dots & & \dots & \dots \\
            C_1 & \vdots & D_{11} & D_{12} \\
            C_2 & \vdots & D_{21} & D_{22}
            \end{bmatrix}}_{G_s}
        \begin{bmatrix}
            x\\
            w\\
            u
        \end{bmatrix}
    \]
\end{minipage}
\hspace{0.5cm}
\begin{minipage}{8cm}
    \centering
    \includegraphics[width=6cm]{bilder/hinf_statespace (2).png}
\end{minipage}

\vspace{0.5cm}

An optimal closed solution for the general system is not known. 
A closed solution for the following special case does exist:
\[
    G_s = \begin{bmatrix}
        A & \vdots & B_1 & B_2 \\
        \dots & & \dots & \dots \\
        C_1 & \vdots & 0 & \begin{bmatrix}0\\I\end{bmatrix} \\
        C_2 & \vdots & \begin{bmatrix}0&I\end{bmatrix} & 0
    \end{bmatrix}
\]
The Controller contains a full state oberver with optimal state feedback.

Among other conditions, the $\mathcal{H}_{\infty}$ problem must have
a solution at $\omega = 0$ and $\omega = \infty$. 
The weighting functions $W$ must ensure this.

\paragraph{Typical Weighting Functinos}
\includegraphics[width=\linewidth]{bilder/rob_hinf_wfunc.png}
