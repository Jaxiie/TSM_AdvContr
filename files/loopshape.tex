\section{Loop Shaping}
Controller design in frequency domain is loop shaping.
The open-loop gain should be shaped, that for a good \(S(s)\) and \(T(s)\).
This can also achieved in the closed-loop with the \(H_\infty\)-control.

\subsection{Loop Gain (open-loop)}
$$L(s):=\ G_{p}(s)G_{c}(s)$$
stability criterion: critical point \(-1\) always on the left in the Nyuist plot.

\subsection{Return Difference}
$$D(s):=1+L(s)=1+\;G_{p}(s)G_{c}(s)$$
large teh is desired

\subsection{Sensitivity}
general we want \(S(s)\) to be small, so external signals have little effects
$$S(s):=\frac{e(s)}{r(s)}=\frac{1}{D(s)}=\frac{1}{1+L(s)}=\frac{1}{1+G_{p}(s)G_{C}(s)}=\frac{y(s)}{d(s)}$$

\subsection{Complementary Sensitivity}
\(T(s)\) should be close to \(1\)
$$T(s):=\frac{y(s)}{r(s)}=\frac{L(s)}{1+L(s)}=\frac{G_{p}(s)G_{c}(s)}{1+G_{p}(s)G_{c}(s)}$$

For optimal tracking \(|T(s)|=1\) for all frequencies, due to limited controller bandwidth this \emph{cannot} be achieved.

\subsection{Specification}

\begin{minipage}{10cm}
    \begin{itemize}
        \item Bandwidth \(\omega_B=0 dB\)
        \item High \(L\) at low frequencies
        \begin{itemize}
            \item good tracking and disturbance rejection
            \item lower margin
        \end{itemize}
        \item Low \(L\) at high frequencies
        \begin{itemize}
            \item robustness against unmodelled dynamics
            \item upper margin
        \end{itemize}
    \end{itemize}
\end{minipage}
\hspace{0.5cm}
\begin{minipage}{8cm}
    \centering
    \includegraphics[width=\linewidth]{./bilder/specL.png}
\end{minipage}

\begin{minipage}{10cm}
    put lower margin around crossover frequency
\end{minipage}
\hspace{0.5cm}
\begin{minipage}{8cm}
    \centering
    \includegraphics[width=\linewidth]{./bilder/specD.png}
\end{minipage}

\begin{minipage}{10cm}
    \begin{itemize}
        \item Low \(|S|\) at low frequencies
        \begin{itemize}
            \item good disturbance rejection
            \item upper margin
        \end{itemize}
        \item Little overshoot at crossover frequency
        \begin{itemize}
            \item upper margin
        \end{itemize}        
    \end{itemize}
\end{minipage}
\hspace{0.5cm}
\begin{minipage}{8cm}
    \centering
    \includegraphics[width=\linewidth]{./bilder/specS.png}
\end{minipage}

\begin{minipage}{10cm}
    \begin{itemize}
        \item Low \(T\) close to \(1\) in low frequency range
        \item Little overshoot at crossover frequency
        \begin{itemize}
            \item upper margin
        \end{itemize}
        \item Steep roll-off at high frequency range
        \begin{itemize}
            \item upper margin
        \end{itemize}    
    \end{itemize}
\end{minipage}
\hspace{0.5cm}
\begin{minipage}{8cm}
    \centering
    \includegraphics[width=\linewidth]{./bilder/specT.png}
\end{minipage}

\subsection{From open-loop to closed-loop}
Low frequency is until \(L\) crosses \(20 dB\)

\includegraphics[width=\linewidth]{./bilder/lowLtoSorT.png}

High frequency is from where \(L\) crosses \(-20 dB\)

\includegraphics[width=\linewidth]{./bilder/highLtoSorT.png}

The overshoot is where \(L\) crosses the unit circle in the Nyquist plot.